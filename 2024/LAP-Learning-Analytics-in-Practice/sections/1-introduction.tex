\section{Introduction}

\begin{frame}{Introduction}
    In response to the increasing demand for advanced learning analytics in educational settings, 
    this project proposes a discussion on ways to enrich the information available about learners 
    in the Moodle learning management system by incorporating external data sources.
    \cite{park2015development}
\end{frame}

\subsection{Context}
\begin{frame}{Context}
    We are constructing a dashboard in Moodle to display the results of predictive models, 
    such as the probability of dropout in a course. 
    To achieve this, we utilize the built-in Moodle Analytics API, 
    which considers indicators as independent variables and the target as the dependent variable.
\end{frame}

\begin{frame}{Context}
    Moodle comes with default indicators and targets, and it is possible to extend its 
    classes to define custom indicators and targets based on Moodle data. 
    However, to increase the model's accuracy, we aim to incorporate external 
    data into Moodle to be used as additional indicators and targets.
\end{frame}

\subsection{Research Questions}
\begin{frame}{Research Questions}
    \begin{enumerate}[<+-|alert@+>]\color{gray}
        \item What kind of external data can be usefully and safely integrated with 
              the Moodle (behavioral) data?
        \item How this enriched data can be used as features for statistical and 
              machine learning models respecting privacy and student autonomy in dashboards?
    \end{enumerate}
\end{frame}