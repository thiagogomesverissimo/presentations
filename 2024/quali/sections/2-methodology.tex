\section{Metodologia}

\begin{frame}{Revisão Terciária}
    Questão de pesquisa formulada para a revisão terciária: 

    \emph{Qual o perfil das revisões sistemáticas existentes com foco em painéis de análise de aprendizagem?}
\end{frame}

\begin{frame}{Revisão Terciária}
  Metodologia de revisão sistemática proposta por \cite{Kitchenham:2007}:
  \begin{itemize}
    \begin{spacing}{0.3}
      \item 1. Planejamento: Definição da Questão de Pesquisa
      \item 2. Planejamento: Desenvolvimento do Protocolo de Revisão
      \item 3. Planejamento: Definição dos critérios de inclusão e de exclusão
      \item 4. Condução Busca Sistemática e Seleção de Estudos
      \item 5. Condução: Extração de Dados
      \item 6. Condução: Síntese e Análise dos Dados
      \item 7. Condução: Avaliação da Qualidade
      \item 8. Resultados: Interpretação dos Resultados
      \item 9. Resultados: Relato dos Resultados
    \end{spacing}
  \end{itemize}
\end{frame}

\begin{frame}{Revisão Terciária}
    \textbf{Cadeia de busca:}

    \textit{"Dashboard" AND "Learning Analytics" AND ("Systematic Literature Review" OR "Systematic Review")}
\end{frame}

\begin{frame}{Revisão Terciária}
    \input{/tmp/tertiary_review_databases_duplicated.tex}
\end{frame}


\begin{frame}{Revisão Terciária}
    {\tiny
    \input{/tmp/tertiary_review_excluded.tex}
    }
\end{frame}

\begin{frame}{Revisão Terciária}
    \begin{figure}[H]
        \centering
        \label{fig:digital_learning}
        %\caption{The data regarding the student learning process is dispersed across multiple sources} 
        \includegraphics[width=0.8\textwidth]{/tmp/tertiary_review_yearly_barplot.pdf}
        \\ \small Quantidade de publicações selecionadas por ano
    \end{figure}
\end{frame}

\begin{frame}{Revisão Terciária}
    {\tiny
    \begin{itemize}
        \begin{spacing}{0.3}
        \item Q1: Quandidade de bases de dados consultadas
        \item Q2: Arcabouço usado na revisão sistemática
        \item Q3: Quantidade de publicações inicialmente selecionadas a partir da cadeia de busca
        \item Q4: Quantidade de publicações resultantes depois de removidos os duplicados 
        \item Q5: Quantidade de publicações resultantes depois de aplicado os critérios de inclusão e exclusão nos títulos e resumos
        \item Q6: Quantidade de artigos finais selecionados da revisão
        \item Q7: Período inical da coleta
        \item Q8: Período final da coleta
        \item Q9: Razão percentual entre Quantidade de artigos finais e Quantidade de publicações inicialmente selecionadas
        \end{spacing}
    \end{itemize}
    }
\end{frame}
  
\begin{frame}{Revisão Terciária}
    {\tiny
        \input{/home/thiago/repos/thiago-doutorado-ime-usp/data/systematic_review/selecionados.tex}
    }
\end{frame}




